\documentclass[margin, 11pt]{res} % Use the res.cls style, the font size can be changed to 11pt or 12pt here

\usepackage{helvet} % Default font is the helvetica postscript font
%\usepackage{newcent} % To change the default font to the new century schoolbook postscript font uncomment this line and comment the one above
\usepackage[colorlinks,
                   linkcolor=black,
                  anchorcolor=black,
                  citecolor=blue,
                  urlcolor=black,
                  CJKbookmarks=true
                  ]{hyperref}
\addtolength{\textheight}{0.5cm}
%\addtolength{\voffset}{-0.1in}
\setlength{\textwidth}{5.2in} % Text width of the document
\linespread{1.1}
\begin{document}
\moveleft.5\hoffset\centerline{\huge\bf Zhuo-Cheng Xiao} % Your name at the top
\moveleft.5\hoffset\centerline{  } % Your address
\moveleft.5\hoffset\leftline {Courant Institute, New York University} % Your address
\moveleft.5\hoffset\leftline{251 Mercer St \#921, New York, NY 10012}
\moveleft.5\hoffset\leftline{Email: \href{mailto:zx555@nyu.edu}{\underline{zx555@nyu.edu}}}
\moveleft.5\hoffset\leftline{Mobile: +1 (520) 312-0434}
\moveleft.5\hoffset\leftline{Home Page: \href{https://sites.google.com/math.arizona.edu/zhuocheng-xiao/home}{\underline{https://sites.google.com/math.arizona.edu/zhuocheng-xiao/home}}}
\moveleft\hoffset\vbox{\hrule width\resumewidth height 1pt}\smallskip % Horizontal line after name; adjust line thickness by changing the '1pt'

\begin{resume}
\section{Employment}
\label{employment}
\textbf{Courant Institute of Mathematical Sciences, \\New York University} \\
Courant Instructor \hfill $09/2021-08/2023$ expected\\
Swartz Fellow \hfill $09/2020-09/2021$ \\
Working with Prof. Lai-Sang Young

\section{Education}
\label{Education}
\textbf{Program in Applied Mathematics, \\The University of Arizona} \hfill $08/2016-08/2020$\\
Ph.D., Applied Mathematics, August 2020\\
Co-Advised by Professors Kevin Lin and Jean-Marc Fellous

\textbf{School of Life Sciences, Peking University, China} \hfill $09/2012-07/2016$\\
Bachelor of Biological Science, July 2016\\
Dual Bachelor of Mathematical Science, 2016 \\
Advised by Prof. Louis Tao

\section{Research Interests}
\label{Research}
The principles of neural computation in brain functions like vision and memory; theoretical and computational methods for statistical dynamics in neural network models; Data-driven modeling in neuroscience.\\
{\bf Specific Research Items}
\begin{itemize}
    \item Neural computation and dynmaics of visual corticies.
    \item Model reduction for complicated dynamics of large neuronal networks.
    \item Non-equilibrium statistical mechanics in neural network models: theory; computational methods.
\end{itemize}

\section{Peer-Reviewed Papers}
\label{Publications1}
\moveleft.5\hoffset\centerline{  } % Your address
\begin{itemize}
\item \textbf{Xiao, Z.}; Lin, K.K; Young, LS. \textit{A data-informed mean-field approach to mapping cortical landscapes.} Under review by PLoS Computational Biology. 

\item \textbf{Xiao, Z.}; Lin, K.K. \textit{Efficiency of Direct and Multilevel Monte Carlo for Spiking Neuron Networks. } Under review by Journal of Computational Neuroscience. 

\item Cai, Y.; Wu, T.; Tao, L.; \textbf{Xiao, Z.$^*$} \textit{Low-Dimensional Manifolds Capture Gamma Oscillations with Model Reduction Methods.} Model Reduction Captures Stochastic Gamma Oscillations on Low-Dimensional Manifolds. \href{Front. Comput. Neurosci. 15:678688.}{\underline{doi: 10.3389/fncom.2021.678688}} (2021)

\item  \textbf{Xiao, Z.}; Lin, K.K.; Fellous, JM. \textit{Conjunctive Reward-Place Coding Properties of Dorsal Distal CA1 Hippocampus Cells.}

\href{https://link.springer.com/article/10.1007/s00422-020-00830-0}{\underline{Biological cybernetics. 2020 Apr;114:285-301.}}


\item \textbf{Xiao, Z.}; Wang, B.; Sornborger, A.; Tao, L. \textit{Mutual Information and Information Gating in Synfire Chains.} \href{https://www.mdpi.com/1099-4300/20/2/102}{\underline{Entropy. 2018, 20(2), 102.}}

\item \textbf{Xiao, Z.}; Zhang, J.; Sornborger, A.; Tao, L. \textit{Cusps enable line attractors for neural computation.} \href{https://journals.aps.org/pre/abstract/10.1103/PhysRevE.96.052308}{\underline{Physical Review E. 2017, 96, 052308.}}

\end{itemize}


\medskip
\section{Manuscripts In-Preparation}
\label{Manuscripts}
\begin{itemize}
\item \textbf{Xiao, Z.}; Lin, K.K; Young, LS. \textit{A data-informed mean-field approach for large scale cortical dynamics.} In preparation. 

\item Wu, T.; Cai, Y.; Tao, L.; \textbf{Xiao, Z.$^*$} \textit{Multi-band neuronal oscillations arise from a Rossler attractor.} In preparation

\item \textbf{Xiao, Z.}; Lin, K.K.; Fellous, JM. \textit{The Dynamics and Reconsolidations of Spatial Representations of Reward in Brain.} In Preparation.

%\item  Dong, Y.; Wang, J.;  \textbf{Xiao, Z.}; Hu, H. \textit{Relief as A Natural Resilience Mechanism Against Depression.} Submitted. (2020)

\item \textbf{Xiao, Z.}; Lin, K.K. \textit{Multilevel Monte Carlo for Spiking Networks.} Submitted. (2020)

\end{itemize}

\section{Permanent Manuscripts}
\begin{itemize}
    \item Wang, C.; \textbf{Xiao, Z.}; Wang, Z.; Sornborger, A.; Tao, L.\textit{A Fokker-Planck approach to graded information propagation in pulse-gated feed-forward neuronal networks.}  \href{https://arxiv.org/abs/1512.00520}{\underline{arXiv:1512.00520.}} (2015)
\end{itemize}

\section{Presentations}
\label{Presentations}
{\bf Conference Talks}
\begin{itemize}
\item \textit{“A data-informed mean-field approach to mapping cortical landscapes”,} 
Society for Mathematical Biology 2021 \hfill $06/2021$
\item \textit{“A data-informed mean-field approach to mapping cortical landscapes”,} 
A Bio Dynamics Days 2021, LMAH-Le Havre Normandie - NYU \hfill $06/2021$
\end{itemize}

{\bf Seminar Talks}
\begin{itemize}
\item \textit{“Model Reduction of Gamma Oscillations”,} 
Modeling and Simulation Group Meeting, NYU \hfill $04/2021$
\item \textit{“Computational Strategies in Analysis of Hippocampal Data”,} 
Analysis and Its Applications Seminar, University of Arizona \hfill $03/2019$
\item \textit{“Multi-Level Monte Carlo Methods for Spiking Networks”,} 
Modeling and Computation Seminar, University of Arizona \hfill $02/2018$
\end{itemize}

{\bf Posters}
\begin{itemize}
\item \textit{“Continuous Reward-Place Coding Properties of Dorsal Distal CA1 Hippocampus Cells”,}
Society for Neuroscience 2019  \hfill $10/2019$
\item \textit{“Multi-Level Monte Carlo Methods for Spiking Networks”,} 
SIAM Conference on Applications of Dynamical Systems (DS19) \hfill $05/2019$
\item \textit{“Multi-Level Monte Carlo Methods for Spiking Networks”,} and \\
\textit{ “Cusps Enable Faithful Information Transfer in Feed-Forward Networks”,}
27th Annual Computational Neuroscience Meeting (CNS 2018) \hfill $07/2018$
\end{itemize}


\section{Teaching \\ Experiences}
{\bf At New York University:}
\begin{itemize}
\item Math 120 {\sl Discrete Mathematics}, Instructor \hfill $2021\;\mathrm{Fa}-2022\;\mathrm{Sp}$ 
\end{itemize}

{\bf At The University of Arizona:}
\begin{itemize}
\item Math 583 {\sl Principles and Methods of Applied \\ Mathematics}, Super TA \hfill $2018\;\mathrm{Fa}-2020\;\mathrm{Sp}$ 
\item Math 254 {\sl Ordinary Differential Equations}, TA \hfill $2017\;\mathrm{Fa}-2018\;\mathrm{Fa}$
\item Math 112 {\sl College Algebra}, Instructor \hfill $2016\;\mathrm{Fa}-2017\;\mathrm{Sp}$
\end{itemize}
Good teaching review in Spring and Fall 2018 for Math 254. 

{\bf At Peking University:}
\begin{itemize}
\item {\sl Mathematical Modeling in the Life Sciences}, TA \hfill $2014\; \mathrm{Sp};2015\; \mathrm{Sp}$
\item {\sl Advanced Mathematics}, TA \hfill $2015\; \mathrm{Sp}$
\item {\sl Journal Club of the Frontier for Life Sciences}, TA \hfill $2014\; \mathrm{Fa}$
\end{itemize}

\section{Review Services}
Plos One

\section{AWARDS AND HONORS}
\label{Awards}
Selected Presentation, $3^{\mathrm{rd}}$ Annual Symposium of Undergraduate Research Honor Program in Biology 2015\\
Best Poster, $2^{\mathrm{nd}}$ Annual Symposium of Undergraduate Research Honor Program in Biology  2014\\
Admitted into Undergraduate Research Honor Program in Biology of Peking University 2013\\
Gold Medal (ranking 5th), $10^{\mathrm{th}}$ Chinese Western Mathematical Olympiad 2010\\

\section{SKILLS AND\\ INTERESTS}
\label{Skills}

Coding Skills for:\\
$\bullet$  Matlab, C, R\\
%Experience for: \\
%$\bullet$  Python, Julia
%----------------------------------------------------------------------------------------

\end{resume}
\end{document}
