\documentclass[margin, 11pt]{res} % Use the res.cls style, the font size can be changed to 11pt or 12pt here

\usepackage{helvet} % Default font is the helvetica postscript font
%\usepackage{newcent} % To change the default font to the new century schoolbook postscript font uncomment this line and comment the one above
\usepackage[colorlinks,
                   linkcolor=black,
                  anchorcolor=black,
                  citecolor=blue,
                  urlcolor=black,
                  CJKbookmarks=true
                  ]{hyperref}
\addtolength{\textheight}{0.5cm}
%\addtolength{\voffset}{-0.1in}
\setlength{\textwidth}{5.2in} % Text width of the document
\linespread{1.1}
\begin{document}
\moveleft.5\hoffset\centerline{\huge\bf Zhuo-Cheng Xiao} % Your name at the top
\moveleft.5\hoffset\centerline{  } % Your address
\moveleft.5\hoffset\leftline {Courant Institute, New York University} % Your address
\moveleft.5\hoffset\leftline{251 Mercer St \#921, New York, NY 10012}
\moveleft.5\hoffset\leftline{Email: \href{mailto:zx555@nyu.edu}{\underline{zx555@nyu.edu}}}
\moveleft.5\hoffset\leftline{Mobile: +1 (520) 312-0434}
\moveleft.5\hoffset\leftline{Home Page: \href{zc-xiao.com}{\underline{zc-xiao.com}}}
\moveleft\hoffset\vbox{\hrule width\resumewidth height 1pt}\smallskip % Horizontal line after name; adjust line thickness by changing the '1pt'

\begin{resume}
\section{Employment}
\label{employment}
\textbf{Courant Institute of Mathematical Sciences, \\New York University} \\
Courant Instructor \hfill $09/2021-08/2023$ expected\\
Swartz Fellow \hfill $09/2020-09/2021$ \\
Working with Prof. Lai-Sang Young

\section{Education}
\label{Education}
\textbf{Program in Applied Mathematics, \\The University of Arizona} \hfill $08/2016-08/2020$\\
Ph.D., Applied Mathematics, August 2020\\
Co-Advised by Professors Kevin Lin and Jean-Marc Fellous \\
Thesis: \href{https://www.proquest.com/docview/2437849732?pq-origsite=gscholar&fromopenview=true}{\textit{Neuronal oscillations: In hippocampal functions and in simulations.}}

\textbf{School of Life Sciences, Peking University, China} \hfill $09/2012-07/2016$\\
Bachelor of Biological Science, July 2016\\
Dual Bachelor of Mathematical Science, 2016 \\
Advised by Prof. Louis Tao

\section{Research Interests}
I use realistic large-scale cortical modeling and interpretable model reduction methods to investigate computation principles in brain functions. I also develop theoretical and computational methods for abstract and realistic cortical models.\\
{\bf Specific Research Items}
\begin{itemize}
    \item Efficient large-scale models of visual cortex.
    \item Interpretable model reductions capturing crucial cortical dynamics in complex models.
    \item Using modern learning methods to map parameters and cortical model dynamics.
\end{itemize}

\section{Publications}
\label{Publications1}
%\moveleft.5\hoffset\centerline{  } % Your address
{\bf Manuscripts}
\begin{itemize}
\item Wu T, Cai Y, Zhang R, Wang Z, Tao L$^*$, \textbf{Xiao, ZC$^*$}. \href{
https://doi.org/10.48550/arXiv.2206.14942}{\textit{Multi-band oscillations emerge from a simple spiking network.}} arXiv preprint arXiv:2206.14942. 2022 Jun 29.

\item \textbf{Xiao, ZC}, Lin KK, Young LS. \href{https://doi.org/10.1371/journal.pcbi.1009718}{\textit{A data-informed mean-field approach to mapping of cortical parameter landscapes.}} PLoS computational biology. 2021 Dec 23;17(12):e1009718.

\item \textbf{Xiao, ZC$^*$}; Lin, KK. \href{https://doi.org/10.1007/s10827-021-00807-3}{\textit{Multilevel monte carlo for cortical circuit models.}}  Journal of Computational Neuroscience. 2022 Feb;50(1):9-15.

\item Cai Y, Wu T, Tao L$^*$, \textbf{Xiao, ZC$^*$};. \href{https://doi.org/10.3389/fncom.2021.678688}{\textit{Model Reduction Captures Stochastic Gamma Oscillations on Low-Dimensional Manifolds.}} Frontiers in Computational Neuroscience. 2021:74.

\item  \textbf{Xiao, Z}, Lin K, Fellous JM. \href{https://link.springer.com/article/10.1007/s00422-020-00830-0}{\textit{Conjunctive reward–place coding properties of dorsal distal CA1 hippocampus cells.}} Biological cybernetics. 2020 Apr;114(2):285-301.

\item \textbf{Xiao, Z}, Wang B, Sornborger AT, Tao L. \href{https://doi.org/10.3390/e20020102}{\textit{Mutual information and information gating in synfire chains.}} Entropy. 2018 Feb 1;20(2):102.

\item \textbf{Xiao, Z}, Zhang J, Sornborger AT, Tao L. \href{https://doi.org/10.1103/PhysRevE.96.052308}{\textit{Cusps enable line attractors for neural computation.}} Physical Review E. 2017 Nov 7;96(5):052308.

\item Wang C, \textbf{Xiao, Z}, Wang Z, Sornborger AT, Tao L. \href{
https://doi.org/10.48550/arXiv.1512.00520}{\textit{A Fokker-Planck approach to graded information propagation in pulse-gated feedforward neuronal networks.}} arXiv preprint arXiv:1512.00520. 2015 Dec 1.
\end{itemize}


\medskip
{\bf Ongoing Work}
\begin{itemize}
\item \textbf{Xiao, ZC}; Lin, KK; Young, LS. \textit{Efficient models of cortical activity via
  local dynamic equilibria and coarse-grained interactions}. In preparation.

\item Zhang R; Wang, Z; Wu, T; Cai, Y; Tao, L$^*$; \textbf{Xiao, ZC$^*$}; Li, Y$^*$. \textit{Learning biological neuronal networks with artificial neural networks: neural oscillations.} In preparation.

%\item  Dong, Y.; Wang, J.;  \textbf{Xiao, Z.}; Hu, H. \textit{Relief as A Natural Resilience Mechanism Against Depression.} Submitted. (2020)

\item \textbf{Xiao, ZC}; Lin, KK. \textit{Multilevel Monte Carlo for Spiking Networks.} Submitted.

\item \textbf{Xiao, ZC}; Lin, KK; Fellous, JM. \textit{The Dynamics and Reconsolidations of Spatial Representations of Reward in Brain.} In Preparation.

\end{itemize}


\section{Presentations}
\label{Presentations}
{\bf Talks}
\begin{itemize}
\item \textit{A data-informed mean-field approach to mapping cortical landscapes,} 
SIAM Annual Meeting \hfill $07/2022$
\item \textit{Efficient models of cortical activity via local dynamic equilibria and coarse-grained interactions}, Courant Instructor Day, NYU \hfill $02/2022$
\item \textit{A data-informed mean-field approach to mapping cortical landscapes,} 
Society for Mathematical Biology \hfill $06/2021$
\item \textit{A data-informed mean-field approach to mapping cortical landscapes,} 
A Bio Dynamics Days, LMAH-Le Havre Normandie - NYU \hfill $06/2021$
\item \textit{Model Reduction of Gamma Oscillations,} 
Modeling and Simulation Group Meeting, NYU \hfill $04/2021$
\item \textit{Computational Strategies in Analysis of Hippocampal Data,} 
Analysis and Its Applications Seminar, University of Arizona \hfill $03/2019$
\item \textit{Multi-Level Monte Carlo Methods for Spiking Networks,} 
Modeling and Computation Seminar, University of Arizona \hfill $02/2018$
\end{itemize}

{\bf Posters}
\begin{itemize}
\item \textit{“Continuous Reward-Place Coding Properties of Dorsal Distal CA1 Hippocampus Cells”,}
Society for Neuroscience 2019  \hfill $10/2019$
\item \textit{“Multi-Level Monte Carlo Methods for Spiking Networks”,} 
SIAM Conference on Applications of Dynamical Systems (DS19) \hfill $05/2019$
\item \textit{“Multi-Level Monte Carlo Methods for Spiking Networks”,} and \\
\textit{ “Cusps Enable Faithful Information Transfer in Feed-Forward Networks”,}
27th Annual Computational Neuroscience Meeting (CNS 2018) \hfill $07/2018$
\end{itemize}


\section{Teaching}
{\bf At New York University:}
\begin{itemize}
\item Math-UA 233 {\sl Theory of Probability}, \hfill $2022\;\mathrm{Fall}$ 
\item Math-UA 262 {\sl Ordinary Differential Equations}, \hfill $2022\;\mathrm{Sping}$ 
\item Math-UA 120 {\sl Discrete Mathematics}, \hfill $2021\;\mathrm{Fall}$
\end{itemize}

{\bf At The University of Arizona:}
\begin{itemize}
\item Math 583 {\sl Principles and Methods of Applied \\ Mathematics}, Super TA \hfill $2018\;\mathrm{Fa}-2020\;\mathrm{Sp}$ 
\item Math 254 {\sl Ordinary Differential Equations}, TA \hfill $2017\;\mathrm{Fa}-2018\;\mathrm{Fa}$
\item Math 112 {\sl College Algebra}, Instructor \hfill $2016\;\mathrm{Fa}-2017\;\mathrm{Sp}$
\end{itemize}
Good teaching review in Spring and Fall 2018 for Math 254. 

%{\bf At Peking University:}
%\begin{itemize}
%\item {\sl Mathematical Modeling in the Life Sciences}, TA \hfill $2014\; \mathrm{Sp};2015\; \mathrm{Sp}$
%\item {\sl Advanced Mathematics}, TA \hfill $2015\; \mathrm{Sp}$
%\item {\sl Journal Club of the Frontier for Life Sciences}, TA \hfill $2014\; \mathrm{Fa}$
%\end{itemize}

\section{Review \\ Services}
\begin{itemize}
    \item PLOS One
    \item NPJ Schizophrenia
    \item Neural Computation
    \item Cognitive Neurodynamics
\end{itemize}

\section{AWARDS}
\label{Awards}
Courant Instructorship, New York University,\hfill 2021-2023\\
Swartz Fellowship in Computational Neuroscience,\hfill 2020-2021\\
Best Presentation, Annual Meeting of Undergraduate Research Honor Program, Peking University\hfill 2015\\
%Best Poster, $2^{\mathrm{nd}}$ Annual Symposium of Undergraduate Research Honor Program in Biology  2014\\
Undergraduate Research Honor Program, Peking University\hfill 2013-2016\\
Ranking 5th, Chinese Western Mathematical Olympiad \hfill 2010\\

\section{SKILLS}
\label{Skills}

Coding Skills for:\\
$\bullet$  Matlab, C, R\\
%Experience for: \\
%$\bullet$  Python, Julia
%----------------------------------------------------------------------------------------

\end{resume}
\end{document}
